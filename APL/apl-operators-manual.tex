\documentclass[11pt]{article}
\usepackage[margin=1in]{geometry}
\usepackage{amsmath,amssymb}
\usepackage{hyperref}
\usepackage{enumitem}
\usepackage[T1]{fontenc}
\usepackage{lmodern}
\title{Alpha-Physical Language (APL)\\Operator's Manual}
\author{APL Working Notes}
\date{v1.0}
\newcommand{\PhiField}{\ensuremath{\Phi}} % structure
\newcommand{\eField}{\ensuremath{e}}      % energy
\newcommand{\pField}{\ensuremath{\pi}}    % emergence
\newcommand{\opamp}{\ensuremath{\,\hat{}\,}} % amplify operator symbol caret
\begin{document}
\maketitle
\begin{abstract}
APL is a compact operator language for describing and biasing regimes in physical and chemical systems.
This manual defines fields, operators, modulation states, syntax, and usage patterns for constructing APL sentences.
\end{abstract}

\section{Fields ("Spirals")}
\begin{itemize}[noitemsep]
  \item \textbf{\PhiField} (Structure): geometry, lattices, boundaries
  \item \textbf{\eField} (Energy): waves, thermodynamics, flows
  \item \textbf{\pField} (Emergence): information, chemistry, biology
\end{itemize}

\section{Universal Operations}
Operators act on fields and machines:
\begin{itemize}[noitemsep]
  \item \texttt{()} Boundary/containment
  \item \texttt{\texttimes} Fusion/convergence/joining
  \item \texttt{\textasciicircum} Amplify/gain
  \item \texttt{\%} Decohere/noise/reset
  \item \texttt{+} Group/aggregation/routing
  \item \texttt{--} Separate/splitting/fission
\end{itemize}

\section{Operator States (UMOL)}
Universal Modulation Operator Law (UMOL):
\begin{itemize}[noitemsep]
  \item \textbf{u} (\(\mathcal{U}\)) Expansion/forward projection
  \item \textbf{d} (\(\mathcal{D}\)) Collapse/backward integration
  \item \textbf{m} (CLT) Modulation/coherence lock
\end{itemize}

\section{Sentence Syntax}
An APL sentence has the form
\begin{center}
\texttt{[Direction][Op] \textbar{} [Machine] \textbar{} [Domain] $\to$ [Regime/Behavior]}
\end{center}
Example: \texttt{u\^{}\textbar{}Oscillator\textbar{}wave}.

\section{Machines and Domains}
\begin{itemize}[noitemsep]
  \item \textbf{Machines:} Oscillator, Reactor, Conductor, Encoder, Catalyst, Filter, ...
  \item \textbf{Domains:} wave, geometry, chemistry, materials, ...
\end{itemize}

\section{Usage Patterns}
\begin{itemize}
  \item Compose operators to encode structure and driving (LHS)
  \item Predict regimes/behaviors (RHS) that should be statistically favored under LHS vs. matched controls
  \item Evaluate with domain-appropriate models and metrics
\end{itemize}

\section{Quick Reference}
\begin{tabular}{ll}
  Field & Meaning \\
  \hline
  \(\Phi\) & structure \\
  \(e\) & energy \\
  \(\pi\) & emergence \\
\end{tabular}

\vspace{1em}
\begin{tabular}{ll}
  Operator & Meaning \\
  \hline
  \texttt{()} & boundary/containment \\
  \texttt{\texttimes} & fusion/joining \\
  \texttt{\textasciicircum} & amplify/gain \\
  \texttt{\%} & decohere/noise \\
  \texttt{+} & group/aggregate \\
  \texttt{--} & separate/split \\
\end{tabular}

\vspace{1em}
\begin{tabular}{ll}
  State & Meaning \\
  \hline
  \texttt{u} & expansion/forward \\
  \texttt{d} & collapse/backward \\
  \texttt{m} & modulation/lock \\
\end{tabular}

\end{document}
