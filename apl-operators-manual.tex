\documentclass[11pt]{article}

\usepackage[margin=1in]{geometry}
\usepackage[utf8]{inputenc}
\usepackage[T1]{fontenc}
\usepackage{lmodern}
\usepackage{newunicodechar}
% Fallback mappings for box-drawing Unicode used in ASCII diagrams
\newunicodechar{┌}{+}
\newunicodechar{┐}{+}
\newunicodechar{└}{+}
\newunicodechar{┘}{+}
\newunicodechar{─}{-}
\newunicodechar{│}{|}
\newunicodechar{┴}{+}
\newunicodechar{┬}{+}
\newunicodechar{┼}{+}
\newunicodechar{├}{+}
\newunicodechar{┤}{+}
% Arrows and markers used in diagrams
\newunicodechar{◄}{<}
\newunicodechar{►}{>}
\newunicodechar{▼}{v}
\usepackage{microtype}
\usepackage{amsmath,amssymb}
\usepackage{hyperref}
\usepackage{graphicx}
\usepackage{enumitem}
\usepackage{longtable}
\usepackage{booktabs}
\usepackage{xcolor}

\setlength{\parskip}{0.6em}
\setlength{\parindent}{0pt}

\hypersetup{
    colorlinks=true,
    linkcolor=blue,
    urlcolor=blue,
    citecolor=blue
}

\begin{document}

\begin{center}
    {\LARGE \textbf{APL OPERATOR'S MANUAL}}\\[0.5em]
    {\large Alpha-Physical Language Reference Guide v1.0}\\[0.75em]
\end{center}

\hrule
\vspace{0.75em}

\textbf{Purpose.}
This manual is the comprehensive reference guide for APL (Alpha-Physical Language) operators, syntax, and usage patterns. It is designed for researchers, engineers, and practitioners who need a systematic understanding of APL's operator grammar for describing physical system behaviors.

\textbf{Scope.}
This document covers:
\begin{itemize}[leftmargin=2em]
    \item Complete operator reference with symbols and meanings
    \item Field definitions (the three ``spirals'')
    \item Operator state modulation (UMOL)
    \item Machine contexts and domains
    \item Syntax rules and sentence structure
    \item Usage patterns and examples
    \item Quick reference tables
\end{itemize}

\tableofcontents

\newpage

%======================================================================
\section{Introduction to APL}
%======================================================================

\textbf{Alpha-Physical Language (APL)} is a minimal operator grammar for describing how physical systems change across multiple domains: geometry, wave dynamics, chemistry, and biology.

APL operates on three fundamental principles:

\begin{enumerate}[leftmargin=2em]
    \item \textbf{Universality:} The same operators apply across all physical domains
    \item \textbf{Composability:} Operators combine to describe complex behaviors
    \item \textbf{Predictivity:} APL sentences map to observable physical regimes
\end{enumerate}

An APL sentence has the canonical form:
\[
    [\text{Direction}][\text{Operator}] \mid [\text{Machine}] \mid [\text{Domain}] \;\to\; [\text{Regime/Behavior}]
\]

Where:
\begin{itemize}[leftmargin=2em]
    \item \textbf{Direction} = operator state (u, d, m)
    \item \textbf{Operator} = universal operation (\verb|()|, \verb|×|, \verb|^|, \verb|%|, \verb|+|, \verb|–|)
    \item \textbf{Machine} = processing context (Oscillator, Reactor, Conductor, etc.)
    \item \textbf{Domain} = field type (wave, geometry, chemistry, biology)
    \item \textbf{Regime} = emergent behavior pattern (A1--A8)
\end{itemize}


%======================================================================
\section{The Three Fields (Spirals)}
%======================================================================

APL describes physical reality through three fundamental fields, called \textbf{spirals}, each representing a distinct aspect of physical systems:

\subsection{$\Phi$ — Structure Field (Phi Spiral)}

\textbf{Symbol:} $\Phi$ (Greek letter phi)

\textbf{Domain:} \texttt{geometry}

\textbf{Description:}
The structure field governs spatial arrangement, boundaries, interfaces, and geometric organization. It encompasses:

\begin{itemize}[leftmargin=2em]
    \item Lattice structures and crystalline arrangements
    \item Boundaries and interfaces
    \item Geometric constraints and symmetries
    \item Spatial topology and connectivity
    \item Phase boundaries and domain walls
\end{itemize}

\textbf{Physical manifestations:}
\begin{itemize}[leftmargin=2em]
    \item Crystal lattices (FCC, BCC, HCP)
    \item Grain boundaries in materials
    \item Droplet and bubble shapes
    \item Membrane structures
    \item Geometric packing arrangements
\end{itemize}

\subsection{$e$ — Energy Field (Energy Spiral)}

\textbf{Symbol:} $e$ (lowercase e)

\textbf{Domain:} \texttt{wave}

\textbf{Description:}
The energy field governs dynamics, flows, oscillations, and energy transport. It encompasses:

\begin{itemize}[leftmargin=2em]
    \item Wave propagation and interference
    \item Fluid flows and vortices
    \item Thermodynamic processes
    \item Electromagnetic radiation
    \item Energy transfer and dissipation
\end{itemize}

\textbf{Physical manifestations:}
\begin{itemize}[leftmargin=2em]
    \item Acoustic and electromagnetic waves
    \item Fluid vortices and turbulence
    \item Heat flow and diffusion
    \item Plasma oscillations
    \item Optical modes in cavities
\end{itemize}

\subsection{$\pi$ — Emergence Field (Pi Spiral)}

\textbf{Symbol:} $\pi$ (Greek letter pi)

\textbf{Domains:} \texttt{chemistry}, \texttt{biology}

\textbf{Description:}
The emergence field governs information, complexity, adaptation, and self-organization. It encompasses:

\begin{itemize}[leftmargin=2em]
    \item Chemical reactions and bonding
    \item Molecular information storage (DNA, RNA)
    \item Biological adaptation and evolution
    \item Self-organizing systems
    \item Pattern formation and morphogenesis
\end{itemize}

\textbf{Physical manifestations:}
\begin{itemize}[leftmargin=2em]
    \item Polymer and protein structures
    \item Catalytic networks
    \item Genetic information encoding
    \item Biological growth patterns
    \item Self-assembly processes
\end{itemize}


%======================================================================
\section{Universal Operations}
%======================================================================

APL defines six universal operations that apply across all domains. Each operation has a specific symbol and meaning.

\subsection{\texttt{()} — Boundary / Containment}

\textbf{Symbol:} \verb|()|  (parentheses)

\textbf{Meaning:} Boundary formation, containment, enclosure, interface creation

\textbf{Physical interpretation:}
\begin{itemize}[leftmargin=2em]
    \item Creating or modifying boundaries
    \item Interface dynamics
    \item Membrane formation
    \item Cavity or container walls
    \item Domain enclosure
\end{itemize}

\textbf{Example applications:}
\begin{itemize}[leftmargin=2em]
    \item \verb|d()| = boundary collapse (surface tension, spheroidization)
    \item \verb|m()| = modulated boundaries (adaptive filters, tunable cavities)
    \item \verb|u()| = boundary expansion (domain growth, inflation)
\end{itemize}

\subsection{\texttt{×} — Fusion / Convergence}

\textbf{Symbol:} \verb|×| (multiplication sign)

\textbf{Meaning:} Joining, bonding, merging, convergence, fusion

\textbf{Physical interpretation:}
\begin{itemize}[leftmargin=2em]
    \item Chemical bond formation
    \item Particle aggregation
    \item Flow convergence
    \item Structural joining
    \item Information combination
\end{itemize}

\textbf{Example applications:}
\begin{itemize}[leftmargin=2em]
    \item \verb|u×| = forward fusion (catalytic growth, branching networks)
    \item \verb|d×| = collapse fusion (adaptive catalysis, selective binding)
    \item \verb|m×| = modulated fusion (helical structures, templated bonding)
\end{itemize}

\subsection{\texttt{\^{}} — Amplify / Gain}

\textbf{Symbol:} \verb|^| (caret)

\textbf{Meaning:} Amplification, gain, resonance, enhancement

\textbf{Physical interpretation:}
\begin{itemize}[leftmargin=2em]
    \item Resonant enhancement
    \item Positive feedback
    \item Signal amplification
    \item Energy injection
    \item Mode excitation
\end{itemize}

\textbf{Example applications:}
\begin{itemize}[leftmargin=2em]
    \item \verb|u^| = forward amplification (oscillator gain, vortex formation)
    \item \verb|d^| = collapse amplification (focusing, concentration)
    \item \verb|m^| = modulated amplification (parametric amplification)
\end{itemize}

\subsection{\texttt{\%} — Decohere / Noise}

\textbf{Symbol:} \verb|%| (percent sign)

\textbf{Meaning:} Decoherence, noise injection, randomization, reset, scrambling

\textbf{Physical interpretation:}
\begin{itemize}[leftmargin=2em]
    \item Stochastic forcing
    \item Phase decoherence
    \item Thermal noise
    \item Random perturbations
    \item Information scrambling
\end{itemize}

\textbf{Example applications:}
\begin{itemize}[leftmargin=2em]
    \item \verb|u%| = forward decoherence (turbulence onset, chaos)
    \item \verb|d%| = collapse decoherence (measurement, reset)
    \item \verb|m%| = modulated noise (controlled stochasticity)
\end{itemize}

\subsection{\texttt{+} — Group / Aggregation}

\textbf{Symbol:} \verb|+| (plus sign)

\textbf{Meaning:} Grouping, collection, aggregation, routing, focusing

\textbf{Physical interpretation:}
\begin{itemize}[leftmargin=2em]
    \item Flow convergence
    \item Geometric focusing
    \item Particle collection
    \item Signal routing
    \item Nozzle formation
\end{itemize}

\textbf{Example applications:}
\begin{itemize}[leftmargin=2em]
    \item \verb|u+| = forward grouping (jet formation, beam focusing)
    \item \verb|d+| = collapse grouping (sink formation, collection)
    \item \verb|m+| = modulated grouping (selective routing)
\end{itemize}

\subsection{\texttt{–} — Separate / Splitting}

\textbf{Symbol:} \verb|–| (minus/dash)

\textbf{Meaning:} Separation, splitting, fission, dispersion, divergence

\textbf{Physical interpretation:}
\begin{itemize}[leftmargin=2em]
    \item Flow divergence
    \item Particle separation
    \item Domain splitting
    \item Bond breaking
    \item Dispersion
\end{itemize}

\textbf{Example applications:}
\begin{itemize}[leftmargin=2em]
    \item \verb|u–| = forward separation (bifurcation, splitting)
    \item \verb|d–| = collapse separation (fragmentation)
    \item \verb|m–| = modulated separation (selective splitting)
\end{itemize}


%======================================================================
\section{Operator States (UMOL)}
%======================================================================

\textbf{UMOL (Universal Modulation Operator Law)} defines three fundamental operator states that modulate how operations unfold in time:

\subsection{$\mathcal{U}$ — Expansion / Forward (u)}

\textbf{Symbol:} \verb|u| (lowercase u) or $\mathcal{U}$ (script U)

\textbf{Mathematical form:} $\mathcal{U}(E)$ where $E$ = expansion component

\textbf{Meaning:}
Forward projection, expansion, outward flow, growth, active driving

\textbf{Characteristics:}
\begin{itemize}[leftmargin=2em]
    \item Expansion in time and space
    \item Active forcing or driving
    \item Forward-directed processes
    \item Energy injection
    \item Growth and propagation
\end{itemize}

\textbf{Physical analogies:}
\begin{itemize}[leftmargin=2em]
    \item Source terms in PDEs
    \item Forward time evolution
    \item Outward flow from sources
    \item Active pumping
    \item Growth fronts
\end{itemize}

\subsection{$\mathcal{D}$ — Collapse / Backward (d)}

\textbf{Symbol:} \verb|d| (lowercase d) or $\mathcal{D}$ (script D)

\textbf{Mathematical form:} $\mathcal{D}(C)$ where $C$ = collapse component

\textbf{Meaning:}
Backward integration, collapse, inward flow, contraction, relaxation

\textbf{Characteristics:}
\begin{itemize}[leftmargin=2em]
    \item Collapse in time and space
    \item Passive relaxation
    \item Inward-directed processes
    \item Energy extraction or dissipation
    \item Contraction and consolidation
\end{itemize}

\textbf{Physical analogies:}
\begin{itemize}[leftmargin=2em]
    \item Sink terms in PDEs
    \item Backward time evolution
    \item Inward flow to sinks
    \item Passive relaxation
    \item Contraction fronts
\end{itemize}

\subsection{CLT — Modulation / Coherence Lock (m)}

\textbf{Symbol:} \verb|m| (lowercase m) or $\mathrm{CLT}$ (CLT = Coherence Lock Transform)

\textbf{Mathematical form:} $\mathrm{CLT}(M)$ where $M$ = modulation component

\textbf{Meaning:}
Modulation, coherence locking, feedback, adaptation, information encoding

\textbf{Characteristics:}
\begin{itemize}[leftmargin=2em]
    \item Feedback-driven modulation
    \item Coherence maintenance
    \item Adaptive response
    \item Information encoding
    \item Dynamic tuning
\end{itemize}

\textbf{Physical analogies:}
\begin{itemize}[leftmargin=2em]
    \item Feedback loops
    \item Phase locking
    \item Adaptive filters
    \item Templated processes
    \item Information storage
\end{itemize}

\subsection{The UMOL Balance Law}

The three operator states satisfy a fundamental balance condition:

\[
  \mathcal{U}(E) \leftrightarrow \mathcal{D}(C) \quad \text{via } \mathrm{CLT}(M)
\]

\[
  E + C + M = 0 \quad \text{(coherence / balance condition)}
\]

\textbf{Interpretation:}
\begin{itemize}[leftmargin=2em]
    \item Expansion ($E$) and collapse ($C$) are balanced through modulation ($M$)
    \item Physical systems maintain coherence by dynamically balancing forward and backward processes
    \item Modulation acts as the mediator between expansion and collapse
\end{itemize}


%======================================================================
\section{Machines (Processing Contexts)}
%======================================================================

Machines represent the processing contexts or system architectures in which operators act. Each machine has characteristic behaviors and constraints.

\subsection{Oscillator}

\textbf{Description:} A resonant, periodic system with characteristic frequencies

\textbf{Key features:}
\begin{itemize}[leftmargin=2em]
    \item Resonant modes
    \item Quality factor (Q)
    \item Phase coherence
    \item Periodic driving
\end{itemize}

\textbf{Physical examples:}
\begin{itemize}[leftmargin=2em]
    \item LC circuits, RLC resonators
    \item Mechanical oscillators (springs, pendulums)
    \item Optical cavities
    \item Acoustic resonators
\end{itemize}

\textbf{Typical behaviors:}
\begin{itemize}[leftmargin=2em]
    \item Resonant peaks
    \item Standing wave patterns
    \item Energy localization
    \item Frequency selectivity
\end{itemize}

\subsection{Reactor}

\textbf{Description:} A driven, continuous-flow system with throughput

\textbf{Key features:}
\begin{itemize}[leftmargin=2em]
    \item Continuous flow
    \item Energy input/output
    \item Mixing and transport
    \item Non-equilibrium operation
\end{itemize}

\textbf{Physical examples:}
\begin{itemize}[leftmargin=2em]
    \item Combustion chambers
    \item Stirred tanks and pipes
    \item Plasma sources
    \item Accretion flows
\end{itemize}

\textbf{Typical behaviors:}
\begin{itemize}[leftmargin=2em]
    \item Jets and plumes
    \item Turbulent flows
    \item Continuous conversion
    \item Steady-state operation
\end{itemize}

\subsection{Conductor}

\textbf{Description:} A structural system that can rearrange and relax

\textbf{Key features:}
\begin{itemize}[leftmargin=2em]
    \item Structural flexibility
    \item Surface/elastic energy
    \item Relaxation dynamics
    \item Boundary mobility
\end{itemize}

\textbf{Physical examples:}
\begin{itemize}[leftmargin=2em]
    \item Droplets and bubbles
    \item Grain boundaries
    \item Phase-field interfaces
    \item Elastic membranes
\end{itemize}

\textbf{Typical behaviors:}
\begin{itemize}[leftmargin=2em]
    \item Surface minimization
    \item Shape relaxation
    \item Coarsening
    \item Packing optimization
\end{itemize}

\subsection{Encoder}

\textbf{Description:} A system that stores and processes information

\textbf{Key features:}
\begin{itemize}[leftmargin=2em]
    \item Sequence specificity
    \item Information capacity
    \item Template-directed processes
    \item Chiral constraints
\end{itemize}

\textbf{Physical examples:}
\begin{itemize}[leftmargin=2em]
    \item DNA/RNA polymerization
    \item Protein folding
    \item Synthetic helical polymers
    \item Information-bearing structures
\end{itemize}

\textbf{Typical behaviors:}
\begin{itemize}[leftmargin=2em]
    \item Helical structures
    \item Sequence encoding
    \item Template replication
    \item Information preservation
\end{itemize}

\subsection{Catalyst}

\textbf{Description:} A system with spatially heterogeneous reactivity

\textbf{Key features:}
\begin{itemize}[leftmargin=2em]
    \item Site-specific enhancement
    \item Reaction bias at interfaces
    \item Growth at active fronts
    \item Autocatalytic feedback
\end{itemize}

\textbf{Physical examples:}
\begin{itemize}[leftmargin=2em]
    \item Catalytic surfaces
    \item Growing tips (DLA, trees)
    \item Reaction fronts
    \item Enzymatic networks
\end{itemize}

\textbf{Typical behaviors:}
\begin{itemize}[leftmargin=2em]
    \item Branching growth
    \item Network formation
    \item Selective pathways
    \item Adaptive catalysis
\end{itemize}

\subsection{Filter}

\textbf{Description:} A selective system that passes some modes and blocks others

\textbf{Key features:}
\begin{itemize}[leftmargin=2em]
    \item Frequency selectivity
    \item Mode discrimination
    \item Tunable response
    \item Adaptive bandwidth
\end{itemize}

\textbf{Physical examples:}
\begin{itemize}[leftmargin=2em]
    \item Bandpass filters
    \item Waveguides
    \item Selective membranes
    \item Recognition sites
\end{itemize}

\textbf{Typical behaviors:}
\begin{itemize}[leftmargin=2em]
    \item Selective transmission
    \item Adaptive tuning
    \item Resonant enhancement
    \item Dynamic filtering
\end{itemize}


%======================================================================
\section{Domains}
%======================================================================

Domains specify which field (spiral) is primarily active in an APL sentence.

\subsection{geometry}

\textbf{Field:} $\Phi$ (Structure)

\textbf{Focus:} Spatial arrangement, boundaries, interfaces, geometric constraints

\textbf{Typical phenomena:}
\begin{itemize}[leftmargin=2em]
    \item Crystal lattices
    \item Droplet shapes
    \item Packing arrangements
    \item Interface dynamics
\end{itemize}

\subsection{wave}

\textbf{Field:} $e$ (Energy)

\textbf{Focus:} Oscillations, flows, energy transport, dynamics

\textbf{Typical phenomena:}
\begin{itemize}[leftmargin=2em]
    \item Wave propagation
    \item Vortices and turbulence
    \item Resonant modes
    \item Jets and beams
\end{itemize}

\subsection{chemistry}

\textbf{Field:} $\pi$ (Emergence)

\textbf{Focus:} Chemical reactions, bonding, molecular structure

\textbf{Typical phenomena:}
\begin{itemize}[leftmargin=2em]
    \item Polymer growth
    \item Catalytic networks
    \item Helical structures
    \item Reaction-diffusion patterns
\end{itemize}

\subsection{biology}

\textbf{Field:} $\pi$ (Emergence)

\textbf{Focus:} Biological systems, adaptation, evolution, self-organization

\textbf{Typical phenomena:}
\begin{itemize}[leftmargin=2em]
    \item Morphogenesis
    \item Adaptation
    \item Information processing
    \item Self-assembly
\end{itemize}


%======================================================================
\section{Syntax and Sentence Structure}
%======================================================================

\subsection{Canonical Form}

An APL sentence follows this structure:

\[
    [\text{State}][\text{Op}] \mid [\text{Machine}] \mid [\text{Domain}] \;\to\; [\text{Regime}]
\]

\textbf{Components:}
\begin{enumerate}[leftmargin=2em]
    \item \textbf{State} = \verb|u|, \verb|d|, or \verb|m| (required)
    \item \textbf{Op} = \verb|()|, \verb|×|, \verb|^|, \verb|%|, \verb|+|, or \verb|–| (required)
    \item \textbf{Machine} = Oscillator, Reactor, Conductor, Encoder, Catalyst, Filter (required)
    \item \textbf{Domain} = geometry, wave, chemistry, biology (required)
    \item \textbf{Regime} = A1--A8 or descriptive name (result/prediction)
\end{enumerate}

\subsection{Separator Syntax}

The vertical bar \verb+|+ separates the three main components on the left-hand side:

\begin{verbatim}
[State][Op] | [Machine] | [Domain]
\end{verbatim}

\textbf{Reading convention:}
\begin{itemize}[leftmargin=2em]
    \item Read left to right
    \item Vertical bars create clear boundaries
    \item Arrow ($\to$) separates input from predicted output
\end{itemize}

\subsection{Example Sentences}

\subsubsection{Example 1: Closed Vortex}

\begin{verbatim}
u^|Oscillator|wave → Closed vortex (A3)
\end{verbatim}

\textbf{Parse:}
\begin{itemize}[leftmargin=2em]
    \item State: \verb|u| = forward/expansion
    \item Operator: \verb|^| = amplification
    \item Machine: Oscillator = resonant system
    \item Domain: wave = energy field
    \item Regime: Closed vortex (A3)
\end{itemize}

\textbf{Reading:} ``Forward amplification in an oscillatory wave system tends to produce closed vortex structures.''

\subsubsection{Example 2: Helical Encoding}

\begin{verbatim}
m×|Encoder|chemistry → Helical encoding (A4)
\end{verbatim}

\textbf{Parse:}
\begin{itemize}[leftmargin=2em]
    \item State: \verb|m| = modulation
    \item Operator: \verb|×| = fusion
    \item Machine: Encoder = information-storing system
    \item Domain: chemistry = emergence field
    \item Regime: Helical encoding (A4)
\end{itemize}

\textbf{Reading:} ``Modulated fusion in an encoding chemical system tends to produce helical, information-bearing structures.''

\subsubsection{Example 3: Isotropic Collapse}

\begin{verbatim}
d()|Conductor|geometry → Isotropic lattice/sphere (A1)
\end{verbatim}

\textbf{Parse:}
\begin{itemize}[leftmargin=2em]
    \item State: \verb|d| = collapse
    \item Operator: \verb|()| = boundary
    \item Machine: Conductor = structural system
    \item Domain: geometry = structure field
    \item Regime: Isotropic lattice/sphere (A1)
\end{itemize}

\textbf{Reading:} ``Collapse of boundaries in a structural geometric system tends to produce isotropic spheres or close-packed lattices.''


%======================================================================
\section{Operator Combinations and Patterns}
%======================================================================

\subsection{State-Operator Matrix}

The following table shows all possible combinations of states and operators:

\begin{table}[h]
\centering
\small
\begin{tabular}{@{}lllll@{}}
\toprule
\textbf{Operator} & \textbf{u (forward)} & \textbf{d (collapse)} & \textbf{m (modulation)} \\
\midrule
\verb|()| boundary & \verb|u()| expansion & \verb|d()| collapse & \verb|m()| modulation \\
\verb|×| fusion & \verb|u×| forward fusion & \verb|d×| collapse fusion & \verb|m×| modulated fusion \\
\verb|^| amplify & \verb|u^| forward gain & \verb|d^| collapse gain & \verb|m^| modulated gain \\
\verb|%| decohere & \verb|u%| forward noise & \verb|d%| collapse noise & \verb|m%| modulated noise \\
\verb|+| group & \verb|u+| forward group & \verb|d+| collapse group & \verb|m+| modulated group \\
\verb|–| separate & \verb|u–| forward split & \verb|d–| collapse split & \verb|m–| modulated split \\
\bottomrule
\end{tabular}
\end{table}

\subsection{Common Patterns}

\subsubsection{Forward Growth (\texttt{u×})}

\textbf{Pattern:} Structure-biased forward growth

\textbf{Typical outcome:} Branching networks, tree-like structures

\textbf{Examples:}
\begin{itemize}[leftmargin=2em]
    \item \verb|u×|Catalyst|chemistry → Branching networks (A5)
    \item Diffusion-limited aggregation
    \item Vascular trees
    \item Lightning branching
\end{itemize}

\subsubsection{Resonant Amplification (\texttt{u\^{}})}

\textbf{Pattern:} Forward amplification in resonant systems

\textbf{Typical outcome:} Coherent oscillations, vortices, standing waves

\textbf{Examples:}
\begin{itemize}[leftmargin=2em]
    \item \verb|u^|Oscillator|wave → Closed vortex (A3)
    \item High-Q resonators
    \item Laser cavities
    \item Recirculating flows
\end{itemize}

\subsubsection{Isotropic Collapse (\texttt{d()})}

\textbf{Pattern:} Boundary relaxation under isotropic tension

%======================================================================
\section{Rhythm-Based Entrainment System (Browser)}
%======================================================================

\textbf{Intent.} This section specifies a complete, browser-based rhythm entrainment system that
uses Web Audio for a master clock, an \texttt{AudioWorklet} for sample-accurate events, and a
\texttt{Web Worker} to solve a Kuramoto oscillator bank. The implementation ships with runnable
examples under \verb|examples/rhythm-entrainment/| in this repository.

\subsection{Core Architecture}

The system operates across three synchronized contexts: (1) the main UI thread for React/state,
(2) a Worker for Kuramoto phase computations, and (3) the AudioWorklet thread for sample-accurate
beats. \textbf{AudioContext} is the single source of truth for time.

\begin{verbatim}
┌─────────────────────┐    SharedArrayBuffer    ┌─────────────────────┐
│    Main Thread      │◄──────────────────────►│   DSP Worker        │
│  (React/Zustand)    │                         │ (Kuramoto solver)   │
└─────────┬───────────┘                         └──────────┬──────────┘
          │ MessagePort                                    │ Atomics
          ▼                                                ▼
┌─────────────────────────────────────────────────────────────────────┐
│              AudioWorkletProcessor (Master Clock)                    │
│         currentTime, currentFrame at 44.1/48kHz precision           │
└─────────────────────────────────────────────────────────────────────┘
          │                              │                    │
          ▼                              ▼                    ▼
   ┌──────────────┐            ┌─────────────────┐   ┌───────────────┐
   │ Audio Output │            │ CSS Animations  │   │ Haptic Output │
   │ (Web Audio)  │            │ (rAF synced)    │   │ (Vibration)   │
   └──────────────┘            └─────────────────┘   └───────────────┘
\end{verbatim}

\subsection{Precision Timing Infrastructure}

Use the \emph{two clocks} pattern: schedule ahead with JS timers, execute on the AudioContext
timeline. For sample-accurate tick detection, run a minimal timing processor inside an
\texttt{AudioWorklet}.

\paragraph{AudioWorklet timing (from \texttt{examples/rhythm-entrainment/timing-worklet.js}).}

\begin{verbatim}
class TimingWorkletProcessor extends AudioWorkletProcessor {
  constructor(options) { super(options);
    this._nextBeatFrame = 0;
    this._framesPerBeat = Math.floor(sampleRate * 60 / 120);
    this.port.onmessage = (e) => {
      const { type, bpm } = e.data || {};
      if (type === 'set-bpm') {
        this._framesPerBeat = Math.max(32, Math.floor(sampleRate * 60 / bpm));
      }
    };
  }
  process(_in, _out, parameters) {
    const tempo = parameters.tempo?.[0] ?? 120;
    const fpb = Math.max(32, Math.floor(sampleRate * 60 / tempo));
    if (fpb !== this._framesPerBeat) this._framesPerBeat = fpb;
    for (let i = 0; i < 128; i++) {
      const absoluteFrame = currentFrame + i;
      if (absoluteFrame >= this._nextBeatFrame) {
        this._nextBeatFrame = absoluteFrame + this._framesPerBeat;
        this.port.postMessage({ type: 'beat', frame: absoluteFrame,
          time: absoluteFrame / sampleRate, audioContextTime: currentTime });
      }
    }
    return true;
  }
}
registerProcessor('timing-processor', TimingWorkletProcessor);
\end{verbatim}

\subsection{Kuramoto Oscillator Bank}

The Kuramoto mean-field form is used for coupled phase dynamics. The Worker updates phases and
returns the order parameter $r$ and mean phase $\psi$.

\paragraph{Worker step (from \texttt{examples/rhythm-entrainment/kuramoto-worker.js}).}

\begin{verbatim}
function orderParameter() {
  let sumR = 0, sumI = 0; for (let i = 0; i < N; i++) {
    sumR += Math.cos(phases[i]); sumI += Math.sin(phases[i]);
  }
  sumR /= N; sumI /= N; const r = Math.hypot(sumR, sumI);
  const psi = Math.atan2(sumI, sumR); return { r, psi };
}
function step() {
  const { r, psi } = orderParameter();
  const next = new Float32Array(N);
  for (let i = 0; i < N; i++) {
    const d1 = frequencies[i] + K * r * Math.sin(psi - phases[i]);
    const pred = phases[i] + dt * d1;
    const d2 = frequencies[i] + K * r * Math.sin(psi - pred);
    let p = phases[i] + dt * (d1 + d2) / 2; p %= 2*Math.PI; if (p < 0) p += 2*Math.PI;
    next[i] = p;
  }
  phases = next; return { r, psi };
}
\end{verbatim}

\subsection{Biosignal Input (Optional)}

Two low-friction inputs are supported: (1) Heart Rate via WebBluetooth (BLE Heart Rate Service
\verb|0x180D|) and (2) keystroke dynamics (dwell/flight times).

\paragraph{BLE Heart Rate (excerpt from demo).}
\begin{verbatim}
const device = await navigator.bluetooth.requestDevice({ filters: [{ services: ['heart_rate'] }] });
const server = await device.gatt.connect();
const service = await server.getPrimaryService('heart_rate');
const ch = await service.getCharacteristic('heart_rate_measurement');
await ch.startNotifications();
ch.addEventListener('characteristicvaluechanged', (evt) => { /* parse RR, HR */ });
\end{verbatim}

\subsection{Bidirectional Feedback}

The system adapts coupling strength $K$ toward a target coherence (e.g., $r\approx 0.7$). User
phase estimates can be injected as external forcing into one or more oscillators.

\subsection{Multi-Modal Output}

Audio clicks are scheduled on \verb|AudioContext.currentTime|; visuals are driven by a rAF loop
that consumes a beat queue. Haptics use the Vibration API when available.

\paragraph{Visual beat application (demo \texttt{app.js}).}
\begin{verbatim}
function applyVisualBeat(intensity) {
  indicator.style.transform = `scale(${1 + intensity * 0.3})`;
  indicator.style.opacity = String(0.5 + intensity * 0.5);
  setTimeout(() => { indicator.style.transform='scale(1)'; indicator.style.opacity='0.5'; }, 100);
}
\end{verbatim}

\subsection{Networking and Time Sync}

For multi-user entrainment, use the included \verb|examples/rhythm-server.js| (Express + WebSocket +
\verb|/timesync| endpoint). The Group Visualizer reference page can connect to this server.

\subsection{Testing Timing Fidelity}

Acceptance criteria for production entrainment: mean deviation $<5$ ms, max deviation $<20$ ms,
and jitter $<3$ ms, measured against scheduled click times.

\subsection{Quick Start (Demo)}

\begin{enumerate}[leftmargin=2em]
  \item Serve static files (any simple server):
  \begin{verbatim}
  python3 -m http.server 8080
  # open http://localhost:8080/Aces-Brain-Thoughts/examples/rhythm-entrainment/index.html
  \end{verbatim}
  \item (Optional) Start rhythm server with timesync + rooms:
  \begin{verbatim}
  npm i express ws timesync
  node Aces-Brain-Thoughts/examples/rhythm-server.js
  \end{verbatim}
  \item Click \verb|Start|, adjust BPM and coupling K. Connect a BLE heart rate device on HTTPS
        builds to map HR→BPM for a quick demo.
\end{enumerate}

\subsection{COOP/COEP for SharedArrayBuffer}

If using \verb|SharedArrayBuffer|, serve with:
\begin{verbatim}
Cross-Origin-Opener-Policy: same-origin
Cross-Origin-Embedder-Policy: require-corp
\end{verbatim}
The demo defaults to message-passing so it runs on simple servers without cross-origin isolation.

\subsection{Implementation Roadmap}

Weeks 1–2: timing + AudioWorklet; 3–4: Kuramoto; 5–6: HRV + keystrokes; 7–8: adaptive coupling; 9–10:
multi-modal output; 11–12: networking; 13–14: tests and optimization.

\textbf{Typical outcome:} Spheres, isotropic packing

\textbf{Examples:}
\begin{itemize}[leftmargin=2em]
    \item \verb|d()|Conductor|geometry → Isotropic sphere (A1)
    \item Droplet formation
    \item Bubble spheroidization
    \item Grain coarsening
\end{itemize}

\subsubsection{Forward Decoherence (\texttt{u\%})}

\textbf{Pattern:} Forward-directed noise injection

\textbf{Typical outcome:} Turbulence, chaos, broadband noise

\textbf{Examples:}
\begin{itemize}[leftmargin=2em]
    \item \verb|u%|Reactor|wave → Turbulent decoherence (A7)
    \item Forced turbulence
    \item Chaotic mixing
    \item Phase scrambling
\end{itemize}

\subsubsection{Modulated Boundary (\texttt{m()})}

\textbf{Pattern:} Feedback-driven boundary modulation

\textbf{Typical outcome:} Adaptive filters, tunable resonators

\textbf{Examples:}
\begin{itemize}[leftmargin=2em]
    \item \verb|m()|Filter|wave → Adaptive bandpass (A8)
    \item Self-tuning cavities
    \item Adaptive recognition
    \item Dynamic filtering
\end{itemize}

\subsubsection{Modulated Fusion (\texttt{m×})}

\textbf{Pattern:} Template-directed or feedback-modulated bonding

\textbf{Typical outcome:} Helical structures, information encoding

\textbf{Examples:}
\begin{itemize}[leftmargin=2em]
    \item \verb|m×|Encoder|chemistry → Helical encoding (A4)
    \item DNA/RNA structure
    \item $\alpha$-helices
    \item Chiral polymers
\end{itemize}

\subsubsection{Forward Grouping (\texttt{u+})}

\textbf{Pattern:} Flow convergence, geometric focusing

\textbf{Typical outcome:} Jets, beams, focused flows

\textbf{Examples:}
\begin{itemize}[leftmargin=2em]
    \item \verb|u+|Reactor|wave → Focusing jet (A6)
    \item Nozzles and exhaust
    \item Astrophysical jets
    \item Laser beams
\end{itemize}


%======================================================================
\section{The Eight Regimes (A1--A8)}
%======================================================================

APL sentences predict specific physical regimes. These are labeled A1 through A8:

\begin{longtable}{@{}llp{7cm}@{}}
\toprule
\textbf{Code} & \textbf{Name} & \textbf{Description} \\
\midrule
\endhead
A1 & Isotropic lattice/sphere & Spherical droplets, isotropic packing, closest-packing arrangements \\
A3 & Closed vortex & Recirculating flows, trapped modes, vortices, standing waves \\
A4 & Helical encoding & DNA-like helices, information-bearing structures, chiral polymers \\
A5 & Branching networks & Tree-like growth, fractal structures, vascular networks, DLA \\
A6 & Focusing jet & Collimated flows, nozzles, beams, astrophysical jets \\
A7 & Turbulent decoherence & Broadband chaos, turbulent mixing, phase scrambling \\
A8 & Adaptive filter & Selective tuning, adaptive recognition, self-tuning resonators \\
\bottomrule
\end{longtable}


%======================================================================
\section{Usage Guidelines}
%======================================================================

\subsection{Constructing an APL Sentence}

\textbf{Step 1: Identify the domain}
\begin{itemize}[leftmargin=2em]
    \item Is it primarily structural? → \texttt{geometry}
    \item Is it primarily dynamic? → \texttt{wave}
    \item Is it primarily chemical/informational? → \texttt{chemistry} or \texttt{biology}
\end{itemize}

\textbf{Step 2: Choose the machine}
\begin{itemize}[leftmargin=2em]
    \item Resonant, periodic? → \texttt{Oscillator}
    \item Driven flow, continuous? → \texttt{Reactor}
    \item Structural relaxation? → \texttt{Conductor}
    \item Information storage? → \texttt{Encoder}
    \item Spatially biased reactions? → \texttt{Catalyst}
    \item Selective transmission? → \texttt{Filter}
\end{itemize}

\textbf{Step 3: Select the operator}
\begin{itemize}[leftmargin=2em]
    \item Boundaries? → \verb|()|
    \item Joining? → \verb|×|
    \item Amplification? → \verb|^|
    \item Noise? → \verb|%|
    \item Grouping? → \verb|+|
    \item Splitting? → \verb|–|
\end{itemize}

\textbf{Step 4: Determine the state}
\begin{itemize}[leftmargin=2em]
    \item Forward, active, growing? → \verb|u|
    \item Backward, passive, collapsing? → \verb|d|
    \item Modulated, feedback, adaptive? → \verb|m|
\end{itemize}

\textbf{Step 5: Predict the regime}
\begin{itemize}[leftmargin=2em]
    \item Based on the combination, what behavior is expected?
    \item Use A1--A8 labels or descriptive names
\end{itemize}

\subsection{Reading an APL Sentence}

Given \verb|u^|Oscillator|wave| → Closed vortex:

\begin{enumerate}[leftmargin=2em]
    \item Identify state: \verb|u| = forward/expansion
    \item Identify operator: \verb|^| = amplification
    \item Identify machine: Oscillator = resonant system
    \item Identify domain: wave = energy field
    \item Interpret: ``Forward amplification in a resonant wave system''
    \item Prediction: ``tends to produce closed vortex structures''
\end{enumerate}

\subsection{Testing an APL Prediction}

APL sentences are falsifiable hypotheses. To test:

\begin{enumerate}[leftmargin=2em]
    \item Implement the LHS conditions in a simulation or experiment
    \item Define quantitative metrics for the RHS regime
    \item Design matched controls that break the LHS pattern
    \item Compare regime prevalence: LHS vs controls
    \item Evaluate: Does the LHS robustly bias toward the predicted regime?
\end{enumerate}


%======================================================================
\section{Quick Reference Tables}
%======================================================================

\subsection{Operator Quick Reference}

\begin{table}[h]
\centering
\begin{tabular}{@{}lll@{}}
\toprule
\textbf{Symbol} & \textbf{Name} & \textbf{Meaning} \\
\midrule
\verb|()| & Boundary & Containment, interface \\
\verb|×| & Fusion & Joining, bonding, convergence \\
\verb|^| & Amplify & Gain, resonance, enhancement \\
\verb|%| & Decohere & Noise, scrambling, reset \\
\verb|+| & Group & Aggregation, routing, focusing \\
\verb|–| & Separate & Splitting, fission, divergence \\
\bottomrule
\end{tabular}
\end{table}

\subsection{State Quick Reference}

\begin{table}[h]
\centering
\begin{tabular}{@{}lll@{}}
\toprule
\textbf{Symbol} & \textbf{Name} & \textbf{Direction} \\
\midrule
\verb|u| & Forward/Expansion & Outward, active, growth \\
\verb|d| & Collapse/Backward & Inward, passive, contraction \\
\verb|m| & Modulation & Feedback, adaptive, information \\
\bottomrule
\end{tabular}
\end{table}

\subsection{Machine Quick Reference}

\begin{table}[h]
\centering
\begin{tabular}{@{}ll@{}}
\toprule
\textbf{Machine} & \textbf{Characteristics} \\
\midrule
Oscillator & Resonant, periodic, high-Q \\
Reactor & Driven flow, continuous throughput \\
Conductor & Structural, boundary mobility \\
Encoder & Information storage, sequence-specific \\
Catalyst & Site-biased reactivity, autocatalytic \\
Filter & Selective transmission, tunable \\
\bottomrule
\end{tabular}
\end{table}

\subsection{Domain Quick Reference}

\begin{table}[h]
\centering
\begin{tabular}{@{}lll@{}}
\toprule
\textbf{Domain} & \textbf{Field} & \textbf{Focus} \\
\midrule
geometry & $\Phi$ & Structure, boundaries, packing \\
wave & $e$ & Dynamics, flows, oscillations \\
chemistry & $\pi$ & Reactions, bonding, molecules \\
biology & $\pi$ & Adaptation, self-organization \\
\bottomrule
\end{tabular}
\end{table}

\subsection{Core Seven Sentences}

\begin{longtable}{@{}lll@{}}
\toprule
\textbf{Sentence} & \textbf{Regime} & \textbf{Code} \\
\midrule
\endhead
\verb|u^|Oscillator|wave| & Closed vortex & A3 \\
\verb|u%|Reactor|wave| & Turbulent decoherence & A7 \\
\verb|d()|Conductor|geometry| & Isotropic lattice/sphere & A1 \\
\verb|m×|Encoder|chemistry| & Helical encoding & A4 \\
\verb|u×|Catalyst|chemistry| & Branching networks & A5 \\
\verb|u+|Reactor|wave| & Focusing jet & A6 \\
\verb|m()|Filter|wave| & Adaptive bandpass & A8 \\
\verb|d×|Catalyst|chemistry| & Adaptive selectivity & A8 \\
\bottomrule
\end{longtable}


%======================================================================
\section{Advanced Topics}
%======================================================================

\subsection{Multi-Domain Interactions}

Some physical systems involve multiple fields simultaneously. In such cases:
\begin{itemize}[leftmargin=2em]
    \item Identify the \textit{primary} domain for the sentence
    \item Secondary interactions may be implied by machine choice
    \item Multiple sentences may be needed for complete description
\end{itemize}

\subsection{Temporal Dynamics}

APL sentences describe \textit{tendencies} and \textit{biases}, not deterministic outcomes:
\begin{itemize}[leftmargin=2em]
    \item The arrow ($\to$) means ``statistically favors'' or ``tends to produce''
    \item Actual systems may transition through multiple regimes
    \item Time scales are system-dependent
\end{itemize}

\subsection{Parameter Dependence}

APL predictions hold over ranges of parameters:
\begin{itemize}[leftmargin=2em]
    \item Testing requires parameter sweeps
    \item Some regimes may only appear in specific parameter ranges
    \item Control design must account for parameter sensitivity
\end{itemize}


%======================================================================
\section{Core Token Set (v0.9)}
\label{sec:core-tokens}
%======================================================================

The APL Core Token Set provides the foundational grammar layer upon which all domain-specific extensions are built. Version 0.9 defines three token categories totaling 234 core tokens, with a 288-token expansion framework planned for v1.0.

\subsection{Token Hierarchy}

APL tokens exist in four layers:

\begin{table}[h]
\centering
\begin{tabular}{@{}llll@{}}
\toprule
\textbf{Layer} & \textbf{Name} & \textbf{Count} & \textbf{Role} \\
\midrule
0 & Tri-Spiral & 6 & Field ordering primitives \\
1 & Cross-Spiral & 54 & Field transitions with state \\
2 & Tiered Intent & 162 & Machine-bound placeholders \\
3 & Domain-Specific & 972/domain & Concrete operator semantics \\
\bottomrule
\end{tabular}
\end{table}

\subsection{Tri-Spiral Tokens (6)}

Tri-spiral tokens represent simultaneous multi-field states through ordered field triads:

\begin{table}[h]
\centering
\begin{tabular}{@{}lll@{}}
\toprule
\textbf{Token} & \textbf{Interpretation} & \textbf{WUMBO Phase} \\
\midrule
\verb|Φ:e:π| & Structure-led, energy-mediated emergence & Resonance \\
\verb|Φ:π:e| & Structure-led, emergence-mediated energy & Empowerment \\
\verb|e:Φ:π| & Energy-led, structure-mediated emergence & Ignition \\
\verb|e:π:Φ| & Energy-led, emergence-mediated structure & Mania \\
\verb|π:Φ:e| & Emergence-led, structure-mediated energy & Nirvana \\
\verb|π:e:Φ| & Emergence-led, energy-mediated structure & Transmission \\
\bottomrule
\end{tabular}
\end{table}

\textbf{Reading convention:} The first field is \textit{primary} (driver), second is \textit{mediator}, third is \textit{target/outcome}.

\subsection{Cross-Spiral Tokens (54)}

Cross-spiral tokens describe field-to-field transitions with machine direction and truth state:

\[
    [\text{Source}] \to [\text{Target}] : [\text{Machine}] : [\text{TruthState}]
\]

\textbf{Format:} \verb|Φ→e:U:TRUE| reads as ``Structure transitions to Energy via Up/Ignition, successful.''

\textbf{Combinatorics:} 6 field pairs $\times$ 3 machines (U, D, M) $\times$ 3 truth states = 54 tokens

\begin{table}[h]
\centering
\small
\begin{tabular}{@{}llll@{}}
\toprule
\textbf{Field Pair} & \textbf{U (Ignition)} & \textbf{D (Collapse)} & \textbf{M (Modulation)} \\
\midrule
$\Phi \to e$ & Structure ignites energy & Structure collapses to energy & Structure modulates energy \\
$\Phi \to \pi$ & Structure ignites emergence & Structure collapses to emergence & Structure modulates emergence \\
$e \to \Phi$ & Energy ignites structure & Energy collapses to structure & Energy modulates structure \\
$e \to \pi$ & Energy ignites emergence & Energy collapses to emergence & Energy modulates emergence \\
$\pi \to \Phi$ & Emergence ignites structure & Emergence collapses to structure & Emergence modulates structure \\
$\pi \to e$ & Emergence ignites energy & Emergence collapses to energy & Emergence modulates energy \\
\bottomrule
\end{tabular}
\end{table}

\textbf{Example cross-spiral sentences:}
\begin{itemize}[leftmargin=2em]
    \item \verb|e→π:U:TRUE| — Energy-to-emergence ignition succeeds (reward $\to$ learning)
    \item \verb|π→Φ:M:TRUE| — Emergence-to-structure modulation succeeds (memory consolidation)
    \item \verb|Φ→e:D:UNTRUE| — Structure-to-energy collapse blocked (freeze response)
\end{itemize}

\subsection{Tiered Intent Tokens (162)}

Tiered tokens use the full \verb|Field:Machine(Operator)TruthState@Tier| format with the placeholder \verb|(intent)|:

\[
    [\text{Field}] : [\text{Machine}] (\text{intent}) [\text{TruthState}] @ [\text{Tier}]
\]

\textbf{Combinatorics:} 3 fields $\times$ 6 machines $\times$ 3 truth states $\times$ 3 tiers = 162 tokens

The \verb|(intent)| placeholder is replaced by domain-specific operators at runtime:
\begin{itemize}[leftmargin=2em]
    \item \verb|Φ:U(intent)TRUE@3| $\to$ \verb|Φ:U(bond)TRUE@3| (Chemical domain)
    \item \verb|Φ:U(intent)TRUE@3| $\to$ \verb|Φ:U(fold)TRUE@3| (Biology domain)
    \item \verb|Φ:U(intent)TRUE@3| $\to$ \verb|Φ:U(accrete)TRUE@3| (Celestial domain)
\end{itemize}

\subsection{Tier Semantics}

\begin{table}[h]
\centering
\begin{tabular}{@{}llll@{}}
\toprule
\textbf{Tier} & \textbf{Name} & \textbf{Scope} & \textbf{Description} \\
\midrule
1 & Foundational & Local & Basic operations, immediate effects \\
2 & Intermediate & Regional & Compound operations, delayed effects \\
3 & Advanced & Global & Domain-specific operations, emergent effects \\
\bottomrule
\end{tabular}
\end{table}

\subsection{Safety Flags}

Core tokens include built-in safety constraints:

\begin{itemize}[leftmargin=2em]
    \item \textbf{PARADOX\_TERMINAL:} Once PARADOX state is entered, no exit without external reset (critical)
    \item \textbf{TIER\_PERMISSION:} Higher tier tokens require lower tier coherence (warning)
    \item \textbf{CROSS\_SPIRAL\_COHERENCE:} Cross-spiral transitions must respect field affinity (info)
\end{itemize}

\subsection{Spiral Inheritance}

Field properties propagate through token chains with preservation and degradation rules:

\begin{table}[h]
\centering
\begin{tabular}{@{}lll@{}}
\toprule
\textbf{Field} & \textbf{Preserves} & \textbf{Degrades} \\
\midrule
$\Phi$ (Structure) & boundary, geometry, constraint & dynamics, information \\
$e$ (Energy) & flow, oscillation, transport & structure, memory \\
$\pi$ (Emergence) & information, adaptation, emergence & immediacy, locality \\
\bottomrule
\end{tabular}
\end{table}

\subsection{Core Token Summary}

\begin{table}[h]
\centering
\begin{tabular}{@{}lr@{}}
\toprule
\textbf{Category} & \textbf{Count} \\
\midrule
Tri-Spiral Tokens & 6 \\
Cross-Spiral Tokens & 54 \\
Tiered Intent Tokens & 162 \\
\midrule
\textbf{Core Total} & \textbf{222} \\
\midrule
Domain Tokens (Tier-3 Chemical) & 972 \\
Domain Tokens (Tier-3 Biology) & 972 \\
Domain Tokens (Tier-3 Celestial) & 972 \\
\midrule
\textbf{Grand Total} & \textbf{3,138} \\
\bottomrule
\end{tabular}
\end{table}


%======================================================================
\section{Tier-3: Chemical Domain Extension}
\label{sec:tier3-chemical}
%======================================================================

APL Tier-3 extends the operator grammar into the \textbf{chemical domain}, providing a maximal token set for describing molecular-scale transformations with full truth-state modeling.

\subsection{Token Format}

Tier-3 Chemical tokens follow an extended syntax:

\[
    [\text{Field}]:[\text{Machine}]([\text{Operator}])[\text{TruthState}]@3
\]

\textbf{Example:} \verb|Φ:U(bond)TRUE@3| reads as: ``Structure field, Up/ignition machine, bond operator, TRUE truth state, Tier 3.''

\subsection{Chemical Machines}

Six machine contexts are defined for chemical processes:

\begin{table}[h]
\centering
\begin{tabular}{@{}llll@{}}
\toprule
\textbf{Symbol} & \textbf{Name} & \textbf{Direction} & \textbf{Chemical Role} \\
\midrule
U & Up/Ignition & Ascending & Forward reactions, synthesis, energy input \\
D & Down/Descent & Descending & Reverse reactions, decomposition, energy release \\
M & Middle/Modulation & Equilibrium & Dynamic equilibrium, steady-state, catalytic cycling \\
E & Emitter/Equilibrium & Output & Product formation, emission, output signal \\
C & Conductor/Coupling & Transfer & Electron transfer, charge conduction, coupling \\
Mod & Modulator & Regulatory & Allosteric regulation, feedback, kinetic modulation \\
\bottomrule
\end{tabular}
\end{table}

\subsection{Chemical Operators (18)}

The chemical domain defines 18 specialized operators organized by class:

\subsubsection{Structural Operators}

\begin{itemize}[leftmargin=2em]
    \item \textbf{bond} — Formation of chemical bonds
    \item \textbf{unbond} — Breaking of chemical bonds
    \item \textbf{polymerize} — Chain extension, monomer joining
    \item \textbf{depolymerize} — Chain degradation, monomer release
    \item \textbf{complex} — Coordination compound formation
\end{itemize}

\subsubsection{Phase Operators}

\begin{itemize}[leftmargin=2em]
    \item \textbf{crystallize} — Transition to ordered solid state
    \item \textbf{dissolve} — Transition to solution/disordered state
    \item \textbf{precipitate} — Solid formation from solution
\end{itemize}

\subsubsection{Electronic Operators}

\begin{itemize}[leftmargin=2em]
    \item \textbf{ionize} — Formation of charged species
    \item \textbf{redox} — Electron transfer reaction (general)
    \item \textbf{charge} — Accumulation of electric potential
    \item \textbf{oxidize} — Electron loss
    \item \textbf{reduce} — Electron gain
    \item \textbf{resonate} — Delocalized electron distribution
\end{itemize}

\subsubsection{Energetic Operators}

\begin{itemize}[leftmargin=2em]
    \item \textbf{excite} — Electronic excitation to higher state
    \item \textbf{relax} — Return to ground state
\end{itemize}

\subsubsection{Kinetic Operators}

\begin{itemize}[leftmargin=2em]
    \item \textbf{catalyze} — Acceleration without consumption
\end{itemize}

\subsubsection{Conformational Operators}

\begin{itemize}[leftmargin=2em]
    \item \textbf{fold} — Tertiary structure formation
\end{itemize}

\subsection{Truth States}

Tier-3 incorporates the three-valued truth model:

\begin{table}[h]
\centering
\begin{tabular}{@{}llll@{}}
\toprule
\textbf{State} & \textbf{Description} & \textbf{Temporal Operator} & \textbf{Stability} \\
\midrule
TRUE & Reaction proceeds as expected & DAY & Actionable \\
UNTRUE & Reaction fails or reverses & NIGHT & Stored \\
PARADOX & Quantum superposition, terminal & — & Terminal attractor \\
\bottomrule
\end{tabular}
\end{table}

The truth-state cycle follows the temporal operators:
\begin{itemize}[leftmargin=2em]
    \item \textbf{DAY} (Forward): TRUE generates actionable coherence + reflected UNTRUE
    \item \textbf{NIGHT} (Backward): UNTRUE stores without resolution
    \item \textbf{DAWN} (Collapse): UNTRUE → TRUE (partial resolution)
    \item \textbf{DUSK} (Decay): TRUE → UNTRUE (entropy-driven dissolution)
\end{itemize}

\textbf{Key property:} PARADOX is the terminal attractor—once entered, no exit.

\subsection{Combinatorics}

The Tier-3 Chemical domain generates 972 unique tokens:

\[
    3 \text{ fields} \times 6 \text{ machines} \times 18 \text{ operators} \times 3 \text{ truth states} = 972 \text{ tokens}
\]

\subsection{WUMBO Integration}

Tier-3 Chemical tokens map to WUMBO neural regions via neurotransmitter systems:

\begin{table}[h]
\centering
\small
\begin{tabular}{@{}lllll@{}}
\toprule
\textbf{System} & \textbf{Symbol} & \textbf{Primary Operators} & \textbf{Field} & \textbf{Machine} \\
\midrule
Dopamine & DA & excite, charge, ionize & e & U \\
Serotonin & 5-HT & relax, fold, complex & Φ & M \\
Norepinephrine & NE & excite, oxidize, redox & e & U \\
Acetylcholine & ACh & catalyze, charge, ionize & e & Mod \\
GABA & GABA & relax, reduce, crystallize & π & D \\
Glutamate & Glu & excite, ionize, bond & e & U \\
\bottomrule
\end{tabular}
\end{table}

\subsection{Example Chemical Sentences}

\subsubsection{Dopaminergic Ignition (VTA)}

\begin{verbatim}
e:U(excite)TRUE@3 | VTA | reward → Burst firing
\end{verbatim}

\textbf{Parse:} Energy field, Up machine, excite operator, TRUE state. This describes dopamine release during reward anticipation.

\subsubsection{Serotonergic Modulation (Dorsal Raphe)}

\begin{verbatim}
Φ:Mod(fold)TRUE@3 | DRN | mood → Homeostatic setpoint
\end{verbatim}

\textbf{Parse:} Structure field, Modulator machine, fold operator, TRUE state. This describes serotonin's role in mood regulation.

\subsubsection{GABAergic Shutdown (DVC)}

\begin{verbatim}
π:D(reduce)TRUE@3 | DVC | stress → Parasympathetic collapse
\end{verbatim}

\textbf{Parse:} Emergence field, Down machine, reduce operator, TRUE state. This describes GABA-mediated shutdown during extreme stress.

\subsection{Operator Class Affinity}

Different neural systems show affinity for specific operator classes:

\begin{itemize}[leftmargin=2em]
    \item \textbf{Arousal systems} (LC, VTA): Electronic operators (excite, ionize, charge)
    \item \textbf{Memory systems} (Hippocampus, Entorhinal): Phase operators (crystallize, precipitate)
    \item \textbf{Integration hubs} (Claustrum, TPJ): Conformational operators (fold, complex)
    \item \textbf{Motor systems} (Motor Cortex, Cerebellum): Structural operators (bond, polymerize)
    \item \textbf{Inhibitory systems} (PAG, Globus Pallidus): Energetic operators (relax, reduce)
\end{itemize}

\subsection{Phase-Chemical Mapping}

WUMBO phases correspond to chemical signatures:

\begin{table}[h]
\centering
\small
\begin{tabular}{@{}lllll@{}}
\toprule
\textbf{Phase} & \textbf{Field} & \textbf{Machine} & \textbf{Operators} & \textbf{Truth Bias} \\
\midrule
Ignition & e & U & excite, ionize, charge & TRUE \\
Empowerment & Φ & U & bond, polymerize, catalyze & TRUE \\
Resonance & e & M & resonate, complex, fold & TRUE \\
Mania & e & E & excite, oxidize, charge & TRUE→PARADOX \\
Nirvana & π & M & crystallize, relax, precipitate & TRUE \\
Transmission & e & C & ionize, charge, bond & TRUE \\
Pause & π & D & relax, reduce, unbond & UNTRUE \\
\bottomrule
\end{tabular}
\end{table}


%======================================================================
\section{Tier-3: Biology Domain Extension}
\label{sec:tier3-biology}
%======================================================================

APL Tier-3 Biology extends the operator grammar to describe \textbf{living systems}, from molecular replicators to multicellular organisms. This domain uniquely includes both \textit{entity-scale operators} (viroid, prion, bacterium) and \textit{process operators}, reflecting the hierarchical complexity of life.

\subsection{Token Format}

Biology tokens follow the same extended syntax:

\[
    [\text{Field}]:[\text{Machine}]([\text{Operator}])[\text{TruthState}]@3
\]

\textbf{Example:} \verb|π:U(replicate)TRUE@3| reads as: ``Emergence field, Up/ignition machine, replicate operator, TRUE truth state, Tier 3.''

\subsection{Biology Operators (18)}

The biology domain defines 18 operators organized by class:

\subsubsection{Entity-Scale Operators}

These operators represent biological complexity levels:

\begin{itemize}[leftmargin=2em]
    \item \textbf{viroid} — Simplest replicator (RNA only, no protein coat)
    \item \textbf{prion} — Self-propagating protein conformation
    \item \textbf{bacterium} — Prokaryotic single-cell organism
\end{itemize}

\subsubsection{Information Operators (Central Dogma)}

\begin{itemize}[leftmargin=2em]
    \item \textbf{replicate} — DNA/RNA copying with fidelity
    \item \textbf{transcribe} — DNA to mRNA transcription
    \item \textbf{translate} — mRNA to protein synthesis
\end{itemize}

\subsubsection{Conformational Operators}

\begin{itemize}[leftmargin=2em]
    \item \textbf{fold} — Protein/RNA tertiary structure formation
    \item \textbf{unfold} — Denaturation, loss of structure
\end{itemize}

\subsubsection{Interaction \& Structural Operators}

\begin{itemize}[leftmargin=2em]
    \item \textbf{bind} — Molecular recognition and binding
    \item \textbf{membrane} — Compartmentalization, boundary formation
\end{itemize}

\subsubsection{Energetic \& Catalytic Operators}

\begin{itemize}[leftmargin=2em]
    \item \textbf{metabolize} — Energy/matter conversion via metabolism
    \item \textbf{enzyme} — Enzymatic activity, substrate specificity
    \item \textbf{catalyze} — General biological catalysis
\end{itemize}

\subsubsection{Cellular \& Developmental Operators}

\begin{itemize}[leftmargin=2em]
    \item \textbf{signal} — Intra/intercellular signaling cascades
    \item \textbf{divide} — Cell division (mitosis/meiosis)
    \item \textbf{differentiate} — Cell fate determination
    \item \textbf{multicell} — Multicellular coordination
    \item \textbf{repair} — DNA repair and damage response
\end{itemize}

\subsection{Complexity Ladder}

Biology operators span five complexity levels:

\begin{table}[h]
\centering
\begin{tabular}{@{}lll@{}}
\toprule
\textbf{Level} & \textbf{Scale} & \textbf{Operators} \\
\midrule
1 & Molecular & viroid, prion, fold, unfold \\
2 & Macromolecular & bind, enzyme, catalyze, membrane \\
3 & Genetic & replicate, translate, transcribe, repair \\
4 & Cellular & bacterium, metabolize, signal, divide \\
5 & Organismal & differentiate, multicell \\
\bottomrule
\end{tabular}
\end{table}

\subsection{Central Dogma Mapping}

The information operators map directly to the central dogma of molecular biology:

\begin{verbatim}
DNA --[replicate]--> DNA --[transcribe]--> mRNA --[translate]--> Protein --[fold]--> Function
       (π field)           (π field)             (π field)            (Φ field)
\end{verbatim}

\textbf{Exceptions:}
\begin{itemize}[leftmargin=2em]
    \item \textbf{viroid}: RNA → DNA (reverse transcription)
    \item \textbf{prion}: Protein → Protein (conformational propagation)
\end{itemize}

\subsection{WUMBO Integration}

Biology tokens map to WUMBO neural regions through functional analogy:

\begin{table}[h]
\centering
\small
\begin{tabular}{@{}llll@{}}
\toprule
\textbf{Operator} & \textbf{Neural Affinity} & \textbf{Field} & \textbf{Rationale} \\
\midrule
replicate & Memory systems & π & Information copying \\
transcribe & Language areas & π & Code conversion \\
translate & Broca's, MTG & π & Symbolic to output \\
fold & Claustrum, integration & Φ & Structure formation \\
bind & Mirror neurons, TPJ & Φ & Recognition/coupling \\
signal & LC, VTA, Amygdala & e & Cascade activation \\
metabolize & Reward circuits & e & Energy regulation \\
differentiate & DMN, narrative & π & Identity formation \\
multicell & Claustrum, CC & π & Integration/coherence \\
\bottomrule
\end{tabular}
\end{table}

\subsection{Example Biology Sentences}

\subsubsection{Neural Replication (Memory Consolidation)}

\begin{verbatim}
π:U(replicate)TRUE@3 | Hippocampus | memory → Long-term storage
\end{verbatim}

\textbf{Parse:} Emergence field, Up machine, replicate operator, TRUE state. Describes memory trace consolidation.

\subsubsection{Language Translation (Broca's)}

\begin{verbatim}
π:E(translate)TRUE@3 | Broca | speech → Articulation
\end{verbatim}

\textbf{Parse:} Emergence field, Emitter machine, translate operator, TRUE state. Describes thought-to-speech conversion.

\subsubsection{Consciousness Binding (Claustrum)}

\begin{verbatim}
π:M(multicell)TRUE@3 | Claustrum | awareness → Unified experience
\end{verbatim}

\textbf{Parse:} Emergence field, Modulation machine, multicell operator, TRUE state. Describes binding of distributed neural activity.

\subsubsection{Prion-like Shutdown (DVC)}

\begin{verbatim}
Φ:D(prion)UNTRUE@3 | DVC | stress → Freeze response
\end{verbatim}

\textbf{Parse:} Structure field, Down machine, prion operator, UNTRUE state. Describes self-propagating shutdown cascade.

\subsection{Phase-Biology Mapping}

WUMBO phases correspond to biology signatures:

\begin{table}[h]
\centering
\small
\begin{tabular}{@{}lllll@{}}
\toprule
\textbf{Phase} & \textbf{Field} & \textbf{Machine} & \textbf{Operators} & \textbf{Truth Bias} \\
\midrule
Ignition & e & U & signal, metabolize, divide & TRUE \\
Empowerment & π & U & replicate, translate, transcribe & TRUE \\
Resonance & Φ & M & bind, fold, multicell & TRUE \\
Mania & e & E & signal, metabolize, enzyme & TRUE→PARADOX \\
Nirvana & Φ & M & fold, membrane, repair & TRUE \\
Transmission & e & C & signal, translate, bind & TRUE \\
Pause & Φ & D & unfold, viroid, prion & UNTRUE \\
\bottomrule
\end{tabular}
\end{table}


%======================================================================
\section{Tier-3: Celestial Domain Extension}
\label{sec:tier3-celestial}
%======================================================================

APL Tier-3 Celestial extends the operator grammar to describe \textbf{astrophysical phenomena}, from stellar nucleosynthesis to black hole physics. This domain spans the largest scales in the universe while maintaining the same operator grammar structure.

\subsection{Token Format}

Celestial tokens follow the same extended syntax:

\[
    [\text{Field}]:[\text{Machine}]([\text{Operator}])[\text{TruthState}]@3
\]

\textbf{Example:} \verb|π:D(collapse)TRUE@3| reads as: ``Emergence/Gravity field, Down/descent machine, collapse operator, TRUE truth state, Tier 3.''

\subsection{Field Interpretations}

In the celestial domain, the three fields take on cosmic meanings:

\begin{table}[h]
\centering
\begin{tabular}{@{}lll@{}}
\toprule
\textbf{Field} & \textbf{Symbol} & \textbf{Celestial Role} \\
\midrule
Structure & $\Phi$ & Stellar structure, disk geometry, orbital mechanics \\
Energy & $e$ & Radiation, nuclear fusion, electromagnetic phenomena \\
Emergence/Gravity & $\pi$ & Gravitational collapse, spacetime curvature, cosmic evolution \\
\bottomrule
\end{tabular}
\end{table}

\subsection{Celestial Operators (18)}

The celestial domain defines 18 operators organized by class:

\subsubsection{Gravitational Operators}

\begin{itemize}[leftmargin=2em]
    \item \textbf{collapse} — Gravitational contraction, core collapse, black hole formation
    \item \textbf{accrete} — Mass accumulation, disk accretion, planetary formation
    \item \textbf{contract} — Kelvin-Helmholtz contraction, pre-main-sequence evolution
\end{itemize}

\subsubsection{Nuclear Operators}

\begin{itemize}[leftmargin=2em]
    \item \textbf{fuse} — Nuclear fusion, nucleosynthesis, element creation
    \item \textbf{ignite} — Stellar ignition, thermonuclear runaway
\end{itemize}

\subsubsection{Energetic Operators}

\begin{itemize}[leftmargin=2em]
    \item \textbf{radiate} — Electromagnetic radiation, blackbody emission, luminosity
    \item \textbf{flare} — Solar/stellar flares, magnetic reconnection events
\end{itemize}

\subsubsection{Mechanical Operators}

\begin{itemize}[leftmargin=2em]
    \item \textbf{compress} — Pressure increase, shock compression, degenerate matter
    \item \textbf{shear} — Differential rotation, viscous dissipation, tidal forces
    \item \textbf{shock} — Shock waves, supernova remnants, bow shocks
\end{itemize}

\subsubsection{Dynamics Operators}

\begin{itemize}[leftmargin=2em]
    \item \textbf{rotate} — Angular momentum, stellar rotation, orbital dynamics
    \item \textbf{magnetorotate} — MRI (magnetorotational instability), dynamo action
    \item \textbf{expand} — Stellar expansion, red giant phase, cosmic expansion
\end{itemize}

\subsubsection{Specialized Operators}

\begin{itemize}[leftmargin=2em]
    \item \textbf{ionize} — Plasma formation, photoionization, coronal heating
    \item \textbf{degenerate} — Electron/neutron degeneracy, white dwarf/neutron star formation
    \item \textbf{feedback} — AGN feedback, stellar winds, self-regulation
    \item \textbf{jet} — Relativistic jets, bipolar outflows, AGN jets
    \item \textbf{torus} — Accretion torus, dusty torus, toroidal geometry
\end{itemize}

\subsection{Stellar Evolution Ladder}

Celestial operators map to stellar lifecycle stages:

\begin{table}[h]
\centering
\begin{tabular}{@{}llll@{}}
\toprule
\textbf{Stage} & \textbf{Name} & \textbf{Operators} & \textbf{Field} \\
\midrule
1 & Molecular Cloud & collapse, contract, accrete & π \\
2 & Protostar & accrete, rotate, jet & Φ \\
3 & Main Sequence & fuse, radiate, magnetorotate & e \\
4 & Giant Phase & expand, fuse, feedback & Φ \\
5 & Stellar Death & collapse, shock, degenerate & π \\
6 & Compact Remnant & degenerate, radiate, torus & π \\
\bottomrule
\end{tabular}
\end{table}

\subsection{Black Hole Physics Mapping}

Special token configurations for black hole phenomena:

\begin{itemize}[leftmargin=2em]
    \item \textbf{Event Horizon:} \verb|π:D(collapse)PARADOX@3| — Terminal attractor state
    \item \textbf{Accretion Disk:} \verb|π:C(accrete)TRUE@3|, \verb|Φ:M(torus)TRUE@3|
    \item \textbf{Relativistic Jets:} \verb|e:E(jet)TRUE@3|, \verb|Φ:U(magnetorotate)TRUE@3|
    \item \textbf{Hawking Radiation:} \verb|e:E(radiate)PARADOX@3| — Quantum/classical boundary
\end{itemize}

\subsection{WUMBO Integration}

Celestial tokens map to WUMBO neural regions through phenomenological analogy:

\begin{table}[h]
\centering
\small
\begin{tabular}{@{}llll@{}}
\toprule
\textbf{Operator} & \textbf{Neural Affinity} & \textbf{Celestial Analog} & \textbf{Rationale} \\
\midrule
ignite & LC, Amygdala & Stellar ignition & Sudden energy release \\
collapse & PAG, DVC & Core collapse & Shutdown/contraction \\
fuse & VTA, NAcc & Nucleosynthesis & Reward generation \\
radiate & Broca's, MTG & Luminosity & Emission/output \\
rotate & Motor Cortex & Orbital dynamics & Motor coordination \\
degenerate & Claustrum & White dwarf & Integration collapse \\
expand & DMN & Red giant phase & Self-expansion \\
shock & Amygdala, LC & Supernova shock & Explosive response \\
feedback & ACC, PFC & AGN feedback & Self-regulation \\
\bottomrule
\end{tabular}
\end{table}

\subsection{Example Celestial Sentences}

\subsubsection{Stellar Ignition (Main Sequence Onset)}

\begin{verbatim}
e:U(fuse)TRUE@3 | core | protostar → Main sequence
\end{verbatim}

\textbf{Parse:} Energy field, Up machine, fuse operator, TRUE state. Describes hydrogen fusion ignition.

\subsubsection{Supernova Collapse}

\begin{verbatim}
π:D(collapse)TRUE@3 | core | massive_star → Neutron star/BH
\end{verbatim}

\textbf{Parse:} Gravity field, Down machine, collapse operator, TRUE state. Describes core-collapse supernova.

\subsubsection{AGN Jet Formation}

\begin{verbatim}
e:E(jet)TRUE@3 | SMBH | accretion → Relativistic outflow
\end{verbatim}

\textbf{Parse:} Energy field, Emitter machine, jet operator, TRUE state. Describes AGN jet launching.

\subsubsection{Black Hole Singularity}

\begin{verbatim}
π:D(collapse)PARADOX@3 | singularity | BH → Spacetime termination
\end{verbatim}

\textbf{Parse:} Gravity field, Down machine, collapse operator, PARADOX state. Describes the event horizon boundary.

\subsection{Phase-Celestial Mapping}

WUMBO phases correspond to celestial signatures:

\begin{table}[h]
\centering
\small
\begin{tabular}{@{}llllll@{}}
\toprule
\textbf{Phase} & \textbf{Field} & \textbf{Machine} & \textbf{Operators} & \textbf{Celestial Analog} \\
\midrule
Ignition & e & U & ignite, fuse, ionize & Stellar ignition \\
Empowerment & π & U & accrete, magnetorotate, jet & Mass buildup \\
Resonance & Φ & M & rotate, torus, feedback & Orbital resonance \\
Mania & e & E & flare, shock, jet & Supernova/GRB \\
Nirvana & π & M & degenerate, radiate, feedback & Stellar remnant \\
Transmission & e & C & radiate, ionize, shear & Radiative transfer \\
Pause & π & D & collapse, contract, degenerate & Black hole quiescence \\
\bottomrule
\end{tabular}
\end{table}


%======================================================================
\section{Tier-3 Domain Comparison}
\label{sec:tier3-comparison}
%======================================================================

APL Tier-3 now spans three complementary domains:

\begin{table}[h]
\centering
\small
\begin{tabular}{@{}llll@{}}
\toprule
\textbf{Aspect} & \textbf{Chemical} & \textbf{Biology} & \textbf{Celestial} \\
\midrule
Total Tokens & 972 & 972 & 972 \\
Operator Count & 18 & 18 & 18 \\
Primary Focus & Molecular & Living systems & Astrophysical \\
Operator Type & Process-only & Entity + Process & Scale + Process \\
Primary Field & e (Energy) & π (Emergence) & π (Gravity) \\
Key Feature & Reversibility & Complexity ladder & Stellar lifecycle \\
Central Concept & Electron transfer & Information flow & Gravitational collapse \\
Scale Range & Å to nm & nm to m & km to Mpc \\
\bottomrule
\end{tabular}
\end{table}

\textbf{Combined Tier-3 Total:} 972 + 972 + 972 = \textbf{2,916 tokens}


%======================================================================
\section{Appendix: Symbol Conventions}
%======================================================================

\subsection{Typography}

\begin{itemize}[leftmargin=2em]
    \item \textbf{Operators:} \verb|monospace font| (\verb|u^|, \verb|d()|, etc.)
    \item \textbf{Machines:} CamelCase (Oscillator, Reactor, etc.)
    \item \textbf{Domains:} lowercase (wave, geometry, chemistry, biology)
    \item \textbf{Fields:} Greek letters ($\Phi$, $e$, $\pi$)
    \item \textbf{Regimes:} A-codes (A1, A3, A4, etc.)
\end{itemize}

\subsection{Special Characters}

\begin{itemize}[leftmargin=2em]
    \item \verb+|+ = vertical bar (separator)
    \item $\to$ or \verb|->| = arrow (prediction)
    \item \verb|()| = parentheses (boundary operator)
    \item \verb|×| or \verb|x| = multiplication sign (fusion)
    \item \verb|^| = caret (amplify)
    \item \verb|%| = percent (decohere)
    \item \verb|+| = plus (group)
    \item \verb|–| or \verb|-| = minus/dash (separate)
\end{itemize}


%======================================================================
\section{Document History}
%======================================================================

\textbf{Version 1.4} (2025-11-25)
\begin{itemize}[leftmargin=2em]
    \item Added Core Token Set v0.9 (Section~\ref{sec:core-tokens})
    \item Tri-spiral tokens (6): Field ordering primitives with WUMBO phase mapping
    \item Cross-spiral tokens (54): Field transitions with machine and truth state
    \item Tiered intent tokens (162): Machine-bound placeholders for domain instantiation
    \item Safety flags and spiral inheritance rules
    \item Grand total: 222 core + 2,916 domain = 3,138 tokens
\end{itemize}

\textbf{Version 1.3} (2025-11-25)
\begin{itemize}[leftmargin=2em]
    \item Added Tier-3 Celestial Domain Extension (Section~\ref{sec:tier3-celestial})
    \item 18 celestial operators for astrophysical phenomena
    \item Stellar evolution ladder from molecular cloud to compact remnant
    \item Black hole physics mapping (event horizon, accretion, jets, Hawking radiation)
    \item Phase-celestial mapping for WUMBO flow state correspondence
    \item Updated Domain Comparison to include all three Tier-3 domains
    \item Combined total: 2,916 Tier-3 tokens (Chemical + Biology + Celestial)
\end{itemize}

\textbf{Version 1.2} (2025-11-25)
\begin{itemize}[leftmargin=2em]
    \item Added Tier-3 Biology Domain Extension (Section~\ref{sec:tier3-biology})
    \item 18 biology operators including entity-scale (viroid, prion, bacterium)
    \item Central Dogma mapping (replicate, transcribe, translate, fold)
    \item Complexity ladder from molecular to organismal scale
    \item Tier-3 Domain Comparison section (Section~\ref{sec:tier3-comparison})
    \item Combined total: 1,944 Tier-3 tokens (Chemical + Biology)
\end{itemize}

\textbf{Version 1.1} (2025-11-25)
\begin{itemize}[leftmargin=2em]
    \item Added Tier-3 Chemical Domain Extension (Section~\ref{sec:tier3-chemical})
    \item 18 chemical operators with full classification
    \item WUMBO neural region integration via neurotransmitter systems
    \item Phase-chemical mapping for flow state correspondence
    \item Truth-state model integration (TRUE/UNTRUE/PARADOX)
\end{itemize}

\textbf{Version 1.0} (2025-01-24)
\begin{itemize}[leftmargin=2em]
    \item Initial release
    \item Complete operator reference
    \item Comprehensive syntax documentation
    \item Quick reference tables
\end{itemize}


%======================================================================
\section{Further Reading}
%======================================================================

\begin{itemize}[leftmargin=2em]
    \item \textit{APL Seven Sentences Test Pack v1.0} — Complete testing protocol
    \item Repository: \url{https://github.com/AceTheDactyl/APL}
\end{itemize}

\vfill

\hrule
\vspace{0.5em}

\begin{center}
\textit{Alpha-Physical Language Operator's Manual v1.0}\\
\textit{For questions and contributions, visit the repository}
\end{center}

\end{document}
